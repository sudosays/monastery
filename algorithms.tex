% Created 2021-03-06 Sat 17:38
% Intended LaTeX compiler: pdflatex
\documentclass[11pt]{article}
\usepackage[utf8]{inputenc}
\usepackage[T1]{fontenc}
\usepackage{graphicx}
\usepackage{grffile}
\usepackage{longtable}
\usepackage{wrapfig}
\usepackage{rotating}
\usepackage[normalem]{ulem}
\usepackage{amsmath}
\usepackage{textcomp}
\usepackage{amssymb}
\usepackage{capt-of}
\usepackage{hyperref}
\author{Julius Stopforth}
\date{\today}
\title{Data Structures and Algorithms}
\hypersetup{
 pdfauthor={Julius Stopforth},
 pdftitle={Data Structures and Algorithms},
 pdfkeywords={},
 pdfsubject={},
 pdfcreator={Emacs 27.1 (Org mode 9.5)}, 
 pdflang={English}}
\begin{document}

\maketitle
\tableofcontents

\section{Summary}
\label{summary}
\href{./about.md}{About these notes} -
\href{./ch01-algorithms.md}{Algorithms} - \href{./ch01-01-graph\_algos.md}{Graph
algorithms} - \href{./ch01-01-dijkstra.md}{Dijkstra} -
\href{./ch01-02-sorting\_algos.md}{Sorting algorithms} -
\href{./ch01-02-heapsort.md}{Heap sort} - \href{./ch02-data\_structures.md}{Data
Structures}
\section{About these notes}
\label{sec:org96bf2f2}
\section{Algorithms}
\label{sec:org9c8dc44}
\subsection{Graph algoritms}
\label{sec:org1c1f1a7}
\subsubsection{Djikstra's Algorithm}
\label{sec:org29b3818}

Finds the shortest distances from a source node in a graphs to all other
nodes in the graph. The algorithm may be stopped early when seeking the
shortest distance between two nodes only.

For the purpose of illustration we will be using the problem of
determining the most effective flights from my hometown airport in
George, South Africa (blue) and Vancouver, Canada (red).

\begin{figure}[htbp]
\centering
\includesvg[width=.9\linewidth]{./images/dsa/flightmap}
\caption{Credit for the base map to \href{https://commons.wikimedia.org/wiki/User:Petr\_Dlouh\%C3\%BD}{Petr Dlouhý}}
\end{figure}

The possible flights we can choose from are listed below as well as a
made up ``ugh'' score (a combination of the flight time and cost).

\begin{center}
\begin{tabular}{llr}
From & To & Ugh\\
\hline
GRJ & CPT & 1\\
GRJ & JHB & 2\\
CPT & DXB & 9\\
CPT & JFK & 22\\
JHB & DXB & 10\\
JHB & LHR & 17\\
DXB & LHR & 8\\
DXB & JFK & 14\\
LHR & YVR & 9\\
JFK & YVR & 8\\
\end{tabular}
\end{center}

\begin{enumerate}
\item Implementations
\label{sec:org3d41452}
\begin{enumerate}
\item Python
\label{python}
\begin{verbatim}
  """
  We could have used a tuple instead but this is for illustrative purposes
  """
  class Flight:
      def __init__(self, destination, ugh):
          self.destination = destination
          self.ugh = ugh

      def __str__(self):
          return "-> %s (%d)" % (self.destination.name, self.ugh)

  class Airport:

      def __init__(self, name, flights=None):
          self.name = name
          self.flights = flights

      def __str__(self):
          output = self.name
          if self.flights is not None:
              output += ":"
              for f in self.flights:
                  output += "\n" + str(f) 

          return output

  def setup():

      YVR = Airport("YVR")
      LHR = Airport("LHR", flights=[Flight(YVR, 9)])
      JFK = Airport("JFK", flights=[Flight(YVR, 8)])
      DXB = Airport("DXB", flights=[Flight(JFK, 14), Flight(LHR, 8)])
      JHB = Airport("JHB", flights=[Flight(LHR, 17), Flight(DXB, 10)])
      CPT = Airport("CPT", flights=[Flight(JFK, 22), Flight(DXB, 9)])
      GRJ = Airport("GRJ", flights=[Flight(CPT, 1), Flight(JHB, 2)])

      return [GRJ, CPT, JHB, DXB, JFK, LHR, YVR]


  def djikstras(airports, start, dest):
      ughs = {}
      prev = {}

      for a in airports:
          ughs[a] = float("inf")
          prev[a] = None

      ughs[start] = 0

      visited = []
      unvisited = ughs

      while len(unvisited) > 0:
          
          current = min({airport:ugh for (airport, ugh) in 
              filter(lambda t: t[0] not in visited, ughs.items())},
              key=ughs.get)

          visited.append(current)

          if current == dest:
              break

          for f in current.flights:
             
              temp = ughs[current] + f.ugh
              
              if temp < ughs[f.destination]:
             
                  ughs[f.destination] = temp
                  prev[f.destination] = current

      return ughs, prev

  if __name__ == "__main__":
      
      airports = setup()
      
      ughs, prev = djikstras(airports, airports[0], airports[-1]) 

      flight_path = airports[-1].name
      current = airports[-1]

      while prev[current] is not None:
          flight_path = prev[current].name + "->" + flight_path
          current = prev[current]

      print(flight_path)
\end{verbatim}
\end{enumerate}
\end{enumerate}

\subsection{Sorting algorithms}
\label{sorting-algorithms}
\subsubsection{Heapsort}
\label{heapsort}
For the purposes of learning the heapsort we will use the scenario of
organising medical case file of a virus outbreak. Due to the immense
strain on resources, these case files have been randomly entered into
the system, but the medical staff now need all of the case files in
chronological order so that they can build an accurate timeline of the
virus spread.

Luckily, the case files are keyed with a 17-digit case reference number
that is constructed using the date, time, and a sequence number.

\begin{verbatim}
  0000 00 00 00 00 00 000

  <year> <month> <day> <hour> <minute> <second> <sequence>
\end{verbatim}

Since the resources available are relatively constrained and it is time
sensitive, we have chosen the heapsort because:
\begin{itemize}
\item it has a running complexity of \(O(n\log{}n)\)
\item it sorts the records in place
\end{itemize}

Using a heapsort is interesting because it leverages a data structure
(the heap) as opposed to an efficient method. Essentially, heapsort
boils down to building the heap (\(O(n)\)), and \(n-1\) heapify
operations (\(O(\log{}n)\) each) to do the final sort.

\begin{enumerate}
\item The Heap
\label{sec:orgcc83b8c}
\item Pseudocode
\label{sec:org8b6ef48}
\item Implementations
\label{implementations}
\end{enumerate}
\subsubsection{Python}
\label{python}
\begin{verbatim}
  class Heap:

      def __init__(self, array, heap_size=-1):
          self.array = array
          self.heap_size = heap_size

      def __str__(self):
          return str(self.array)

      def left_of(index):
          return (2*index + 1)

      def right_of(index):
          return (2*index + 2)

      def _swap(self, a, b):
          self.array[a], self.array[b] = self.array[b], self.array[a]

      def _max_heapify(self, index):
          left = Heap.left_of(index)
          right = Heap.right_of(index)

          largest = index

          if left < self.heap_size:
              if self.array[left] > self.array[largest]:
                  largest = left

          if right < self.heap_size:
              if self.array[right] > self.array[largest]:
                  largest = right

          if largest != index:
              self._swap(largest, index)
              self._max_heapify(largest)
          else:
              return

      def _build_max_heap(self):
          self.heap_size = len(self.array)

          for i in range(self.heap_size//2, -1, -1):
              self._max_heapify(i)

      def sort(self):
          self._build_max_heap()

          for i in range(len(self.array)-1,0,-1):
              self._swap(i,0)
              self.heap_size -= 1
              self._max_heapify(0)

  if __name__ == "__main__":

      case_numbers = [
          20200127223412001,
          20200212000201001,
          20200128141212003,
          20200205130501001,
          20200204221331001,
          20200126161212001,
          20200206184812001,
          20200131200531002,
          20200128134516001,
          20200125103945001,
          20200128134612002,
          20200131144454001
      ]

      h = Heap(case_numbers)

      h.sort()

      print(h)
\end{verbatim}


\section{Data Structures}
\label{sec:org8f4f571}
\end{document}
