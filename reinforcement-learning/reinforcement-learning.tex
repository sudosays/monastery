\documentclass[12pt,notitlepage]{article}

\usepackage{hyperref}
\hypersetup{colorlinks = true, linkcolor = blue}

\usepackage{graphicx}
\graphicspath{{images/}}

% To use glossaries uncomment below

%\usepackage[toc]{glossaries}
%\makeglossaries
%\include{<glossaryfilename>}

\begin{document}

\title{Reinforcement Learning}
\author{}
\date{}
\maketitle

%\tableofcontents
%\newpage

These notes are made from studying the book by~\cite{sutton2018reinforcement}.
More specifically, the reading path is Chapters 1--3, 6, 8--10. Furthermore, I
have included a glossary of terms as they are used for reference as well as a
symbol reference if necessary.

\section{Overview}

The aim of a reinforcement learning agent is to learn how to map situations to
actions in order to maximise a numerical~(?) reward signal. This assumes that
the agent has a direct sensorimotor connection to its environment through which
it can sense its environment~(to an extent) and take action that directly
affects the state of said environment. In order to learn this sensorimotor
mapping, the agent typically uses trial-and-error in conjunction with the
reward signal in order to learn which actions or sequence of actions for an
observed state will yield the greatest reward over time. This reward
maximisation is approached by balancing two methods: exploration, and
exploitation. With exploration, the agent tries new actions in different states
to seek out more rewarding actions while with exploitation, the agent will
strongly prefer the most reliably rewarding action so far.

In terms of the greater picture of learning methods, reinforcement learning is
neither a supervised or unsupervised learning method. However, sometimes RL
systems may incorporate supervised or unsupervised learning methods within them
as sub-systems to contribute to the overall sensation or actuation~(?) of the
agent. For example, agents that incorporate a camera to sense their environment
may use a image understanding neural network that was trained using supervised
learning. %Add better explained examples perhaps a diagram.

RL tackles the problems it faces in a holistic
way~(?) as agents are goal-seekers in an interactive, but uncertain
environment. This means that one of the strengths of RL lies in the agent's
ability to learn from its own experience despite operating in an environment
outside of it's experience or previously observed sets of states. This ability
to adapt in an uncertain environment is one of the main strengths and aims of
Reinforcement Learning. However, as a goal-seeker the design~(?use) of explicit
goals is imperative as it allows the agent to better judge its progress toward
that goal.

It is important to note that when referring to reinforcement learning this can
mean one of three things:
\begin{enumerate}
    \item A problem~(?)
    \item Class of solution methods that work well on the problem
    \item Field that studies the problem and its solution methods
\end{enumerate}

In these notes, the terms are differentiated by referring to the field as just
Reinforcement Learning the capitalised term, solutions as algorithms or agents,
and problems as tasks or problems. The lowercase term ``reinforcement
learning'' applies generally to all three definitions while prefixing the RL
before a term is the same. 

\section{Elements of RL}

While the agent and the environment are the key objects for RL, there are also
four different elements that enable an RL learning system to function:
\begin{description}
    \item[policy] is the mapping of states to actions to be perfomed in that
        state
    \item[reward signal] is the response of the environment to the agent for a
        given action in a state
    \item[value function] is a way of measuring the long-term desirability of
        states taking into account the policy~(?)
    \item[model] of the environment is an optional representation that can be
        used for planning
\end{description}

The above definitions are quite broad and should only give one a sense of what
their purpose is in RL an are not complete definitions.

\subsection{Policy}

Most important is the policy (typically denoted by $\pi$) as it alone is
sufficient to determine behaviour of an agent. Furthermore, policies may be
``good'' or ``bad'' in that certain sequences of actions will ultimately yield
less overall reward than others. This is why RL algorithms typically perform
some kind of policy search in order to determine the optimal policy for an
agent. Finally, a policy may also be stochastic in that it defines a
probability distribution of actions to choose given a specific state. As
opposed to a deterministic policy wherein the agent says ``given my current
situation I will always choose this action'', an agent maintaining a
stochastic policy says, ``given my current situation I will most probably do this''.

%\begin{figure}[ht]
%    \centering
%    \def\svgwidth{0.5\columnwidth}
%    \input{images/stochasticpolicyvis.pdf_tex}
%    \caption{Probability distribution for actions and states}\label{fig:stochasticpolicyvis}
%\end{figure}

\subsection{Reward vs. Value}

The terms \emph{reward signal} and \emph{value function} may seem closely
related, but are fundamentally different in how they are incorporated into RL
algorithms. The reward signal defines the goal of a RL problem and is the
primary basis for altering the policy. Since the agent attempts to maximise the
long-term future reward, it builds an idea of the value attached to a
particular state it finds itself in. In this sense, rewards are the immediate,
short-term gains while the value function attempts to determine the long-term
rewards possible. Therefore, it makes sense for the agent to take the most
valuable action even if it is not the most rewarding immediately. It is a sense
of delayed gratification for the agent as opposed to greedily chasing reward
rich actions that might not pan out in the long run. Furthermore, while rewards
are given directly by the environment, values are estimated and updated by the
agent based on historical observations over its lifetime. As with policy
search, value estimation is crucial in most RL algorithms with the aim to do so
accurately and efficiently.

\subsection{Models}

Broadly speaking, a model is a representation of some system that allows for
the inference and prediction of the system to some degree of accuracy. For
example, when Nicolaus Copernicus proposed the heliocentric model of the solar
system, he was able to predict to a fair degree of accuracy the paths and
orbits of the planets and when they would next appear in the night sky
\emph{[CITATION NEEDED]}. With RL, a model can be used by an agent in order to
plan actions or predict the possible result of a sequence of actions before
performing it. However, the use of a model is optional. Methods involving a
model are referred to as \emph{model-based} methods while those who do not are
referred to as \emph{model-free} methods. Furthermore, there exists an entire
spectrum of methods from model-free to model-based including guided policy
search (citation needed) (more examples needed). Typically, for problems in
which no model exists or a model is too computationally expensive to use,
model-free methods are used. For problems where accuracy matters, or the model
is well known, model-based methods can be used for expedited learning.

%\appendix

\bibliography{refs.bib}

\end{document}
